\documentclass[11pt]{scrreprt}
\usepackage[utf8]{inputenc}
\begin{document}

\begin{center}
\Huge
VLV.ical

\medskip
\large
Javascript und Browser-Plugin zur \"Ubertragung des VLV in den eigenen Kalender

\vskip 1in
\Large
Review-Dokument der ersten Phase

\large
\vskip 2in
Softwareprojekt Sommersemester 2015

\medskip
\textit{
Ende der ersten Phase "Analyse- und Entwurf" am 07. April 2015
}
\vskip 3in

\normalsize
\textup{ 
Norris Sam Osarenkhoe \qquad
Florian Raetz \qquad
}
\textup{ 
Erik Langenhan \qquad
Cem Kaygusuz \qquad
}
\textup{ 
Rajeethan Dhayaparan \qquad
Berk Bayraktar \qquad
}
\textup{ 
Michael H\"ahnel \qquad
Tim Reinhold
}
\end{center}

\newpage
\tableofcontents

\newpage
\chapter{Einleitung}
\section{Notwendigkeit der Software}

\normalsize \normalfont \textnormal
Der Stundenplan der Studierenden der TU Ilmenau wird in einem online Vorlesungsverzeichnis veröffentlicht. In diesem sind die einzelnen Vorlesungen, Übungen oder externe Veranstaltungen der Studenten nach dem jeweiligen Semester aufgelistet. Die Übertragung dieser Veranstaltungen auf externe Geräte ist bisher nur durch manuelle Eingabe in andere Kalender möglich. Da hierfür noch keine andere benutzerfreundlichere Lösung gefunden wurde, führte es dazu, dass dieses Problem nun in einem Softwareprojekt von uns analysiert und bearbeitet werden soll.  
Um die Arbeit der manuellen Übertragung dem Anwender abzunehmen, soll es ihm möglich sein durch ein Browser-Plug-in die Termine als ics-Datei zu exportieren und in seinen eigenen Kalender zu importieren. Dies wiederum erspart dem Benutzer Zeit und mühseliges eintippen der Termine in seinen Kalender.

\section{Beschreibung des Aufgabenprojekts}

\section{Muss-Kriterien}
\begin{itemize}
\item \textbf{M1}: Oberfl\"achenlayout-Erweiterung der Software mittels Browser-Extension in Google Chrome
    \begin{itemize}
    \item Die urspr\"ungliche Oberfläche des Vorlesungsverzeichnisses wird unter anderem um Auswahl- und Downloadfunktionen erweitert, um die Interaktion mit dem Nutzer zu erm\"oglichen.
    \item Durch farbliche Untermalung werden dem benutzer ausgew\"ahlte Elemente visualisiert.
    \item Ein zus\"atzliches Popup Fenster erm\"oglicht die Konfiguration der Erweiterung.
    \end{itemize}
\item \textbf{M2}: Auslesen der Veranstaltungsdaten
    \begin{itemize}
    \item Das Auslesen der Veranstaltungsdaten erfolgt mittels Javascript und Regular Expressions.
    \item Auszulesende Daten sind im einzelnen der Veranstaltungsname, der/die Lesende(r), die Uhrzeit (inklusive Dauer), der Zeitraum (angegeben in Kalenderwochen oder einzelnen Daten) und der Ort.
    \end{itemize}
\item \textbf{M3}: Download einer nach dem vCalender Standard formatierten Datei (Endung: *.ics)
    \begin{itemize}
    \item Es soll dem Benutzer m\"oglich sein, angezeigte Veranstaltungen mit wenigen Klicks in einer fertig konfigurierten *.ics Datei zu speichern, um diese im Anschluss in beliebe Kalenderapplikationen zu importieren.
    \end{itemize}
\item \textbf{M4}: Modifizierung der Website zur besseren Lesbarkeit\textbackslash Bedienung
    \begin{itemize}
    \item Die Verwendung von festen div-Containern oder die fehlende Abgrenzung von einzelnen Elementen innerhalb der Website erschwert dem Betrachter die \"Ubersicht \"uber die Informationen. Um eine \"ubersichtliche Struktur zu gew\"ahrleisten, werden einzelne Elemente voneinander visuell abgetrennt und die wichtigen Informationen hervorgehoben.
    \end{itemize}
\end{itemize}

\section{Wunsch-Kriterien}
\begin{itemize}
\item \textbf{W1}: Erweiterung für andere (Desktop-) Browser portieren
    \begin{itemize}
    \item Nach der Fertigstellung der Erweiterung für Google Chrome könnte eine Portierung auf andere Browser stattfinden.
    \end{itemize}
\item \textbf{W2}: visuelle Hervorhebung der zuletzt aktualisierten Veranstaltungen
    \begin{itemize}
    \item Um dem Benutzer des Vorlesungsverzeichnisses auf \"Anderungen in der Veranstaltungsliste hinzuweisen würde eine visuelle Hervorhebung dieser Veranstaltung ihn darüber informieren, dass in der jeweiligen Veranstaltung k\"urzlich \"Anderungen vorgenommen wurden.
    \end{itemize}
\item \textbf{W3}: Abstraktion des Quellcodes, um eine Anwendung auch außerhalb des VLV zu bieten
    \begin{itemize}
    \item Eine Abstraktion
    \end{itemize}
\end{itemize}

\section{Abgrenzungskriterien}
\begin{itemize}
\item Die Software bietet ausschließlich den Download der angezeigten Veranstaltungen, lediglich die Übungen sollen ausgewählt werden können.
\item Die erzeugte *.ics Datei ist nur mit wenigen Kalenderapplikationen kompatibel.
\end{itemize}

\chapter{Verwendete Technologien und Entwicklungswerkzeuge}
\section{Codeverwaltung}
\paragraph{Git \& GitHub}
Der Sourcecode des Projektes wird über die Versionierungssoftware Git verwaltet, welches wiederum über das Portal GitHub.com für jeden als freie Software zur Verf\"ugung gestellt wird.
Git ermöglicht es allen, gleichzeitig an der Software zu arbeiten und die \"Anderungen im Anschluss wieder zusammen zu f\"uhren um eine ganzheitliche Software zu erstellen. Sie bietet außerdem eine vollst\"andige History, womit Fehler zu jeder Zeit r\"uckg\"angig gemacht werden k\"onnen.
Des weiteren ist vollst\"andig nachvollziehbar, welche Person zu welchem Zeitpunkt an welchem St\"uck Quellcode gearbeitet hat.

\section{Kommunikation und Organisation}
\paragraph{Gitter.com}
Zur Kommunikation wird haupts\"achlich Gitter genutzt, welches sehr stark mit GitHub.com verkn\"upft ist und somit eine sehr stark projektorientierte Kommunikationsplattform. Sie bietet neben spezieller Textformatierung f\"ur Quellcode auch die Referenzierung auf Personen, Tickets und auch Dateien im Github Projekt. Außerdem sind auf der Plattform aktuelle Aktivit\"aten des Projektes \"uber sogenannte "Git-Hooks" ersichtlich.
\paragraph{WhatsApp}
Zur Organisation von Terminen, Versp\"atungen u.ä. dient außerdem die bekannte Messaging-Applikation WhatsApp.
\paragraph{Trello}
Zur Aufgabenverteilung, -zuweisung und auch zur Fortschrittsanalyse wird das Online-Portal Trello.com genutzt. Durch eine Browser Erweiterung bietet es außerdem eine Unterstützung des Vorgehensmodells SCRUM.

\section{Programmiersprache}
\paragraph{Javascript}
Da es sich bei diesem Projekt um eine Browser Erweiterung handelt, wird hauptsächlich die Programmiersprache Javascript zum Einsatz kommen.
\paragraph{Bibliotheken}
Um die grundlegenden Möglichkeiten von Javascript zu erweitern werden wir uns an freien Javascript Bibliotheken, wie z.B. jQuery und möglicherweise noch anderen bedienen.

\chapter{Glossar}

\end{document}
