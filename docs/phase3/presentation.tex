\documentclass[10pt]{beamer}
%\usetheme{Ilmenau}
\usepackage[utf8x]{inputenc}
\renewcommand\sfdefault{phv}
\renewcommand\familydefault{\sfdefault}
\usetheme{default}
\usepackage{color}
\useoutertheme{default}
\usepackage{texnansi}
\usepackage{marvosym}
\definecolor{bottomcolour}{rgb}{0.32,0.3,0.38}
\definecolor{middlecolour}{rgb}{0.08,0.08,0.16}
\setbeamerfont{title}{size=\Huge}
\setbeamercolor{structure}{fg=gray}
\setbeamertemplate{frametitle}[default]%[center]
\setbeamercolor{normal text}{bg=black, fg=white}
\setbeamertemplate{background canvas}[vertical shading]
[bottom=bottomcolour, middle=middlecolour, top=black]
\setbeamertemplate{items}[circle]
\setbeamerfont{frametitle}{size=\huge}
\setbeamertemplate{navigation symbols}{} %no nav symbols
\usepackage{amsmath}
\usepackage{amsfonts}
\usepackage{amssymb}
\usepackage{graphicx}
\author{Berk Bayraktar\quad Norris Sam-Osarenkhoe\quad Erik Langenhan}
\title[VLV.ical]{VLV.ical - Kalenderexport für das VLV}

\date{\today} 

\begin{document}

\frame{\titlepage}

\begin{frame}
    \frametitle{Einleitung}
\end{frame}

\begin{frame}
    \frametitle{Zielsetzung}
\end{frame}

\begin{frame}
    \frametitle{Technischer Hintergrund (nur kurzer Anriss)}
\end{frame}

\begin{frame}
    \frametitle{UI}
\end{frame}

\begin{frame}
    \frametitle{Veröffentlichung}
    Quellcode: github.com/vlvical/vlv.ical \\
    Website: vlvical.github.io (o.ä.)
\end{frame}

\end{document}
