\documentclass[11pt]{scrreprt}
\usepackage[utf8]{inputenc}
\usepackage{graphicx}
\begin{document}

\begin{center}
\Huge
VLV.ical

\medskip
\large
Javascript und Browser-Plugin zur \"Ubertragung des VLV in den eigenen Kalender

\vskip 1in
\Large
Gruppe 14 \\
Review-Dokument der zweiten Phase

\large
\vskip 2in
Softwareprojekt Sommersemester 2015

\medskip
\textit{
Statusbericht über die Implementierungsphase
}
\vskip 3in

\normalsize
\textup{ 
Norris Sam Osarenkhoe \qquad
Florian Raetz \qquad
}
\textup{ 
Erik Langenhan \qquad
Cem Kaygusuz \qquad
}
\textup{ 
Rajeethan Dhayaparan \qquad
Berk Bayraktar \qquad
}
\textup{ 
Michael H\"ahnel \qquad
Tim Reinhold
}
\end{center}
%% Kopfblatt mit allgemeinen Informationen zum zweiten Reviewdokument
\newpage

\chapter{Aktueller Stand des Projektes}
&& Wo das dann einzuordnen ist, ist noch zu klären.
\section{Visueller Stand:}
\normalsize \normalfont \textnormal
Die Vorlesungsverzeichniswebsite wurde optisch optimiert, dazu gehört das wegfallen von überlüssigen leeren Balken und die Hervorhebung der einzelnen Veranstaltungen. Hierbei wurden die einzelnen Veranstaltungsblöcke durch blaue Rahmen erkennbar abgetrennt und die jeweiligen Titel in ausgefüllten Kästchen hervorgehoben. Zusätzlich wurde in der Website in der oberen rechten Ecke ein "VlV.ical"-Button hinzugefügt. Durch anklicken dieses Buttons öffnet sich eine Randspalte rechts auf der die ausgewählten Veranstaltungen dargestellt sind. In dieser Spalte befinden sich außerdem die Buttons "click here to hide" (rot), "Download selected" (grün) und "Download All" (weiß). Nach anklicken einer ausgewählten Veranstaltung in der Randspalte erscheint ein Pop-up-Fenster, in der die genauen Details der Veranstaltung dargestellt sind.

\section{Technischer Stand:}
\normalsize \normalfont \textnormal
...

\section{Fehlende/Fehlerhafte Anforderungen:}
\normalsize \normalfont \textnormal
...
